\documentclass[
	% -- opções da classe memoir --
	12pt,				% tamanho da fonte
	openright,			% capítulos começam em pág ímpar (insere página vazia caso preciso)
	oneside,			% para impressão em verso e anverso. Oposto a oneside
	a4paper,			% tamanho do papel. 
	% -- opções da classe abntex2 --
	chapter=TITLE,		% títulos de capítulos convertidos em letras maiúsculas
	%section=TITLE,		% títulos de seções convertidos em letras maiúsculas
	subsection=TITLE,	% títulos de subseções convertidos em letras maiúsculas
	%subsubsection=TITLE,% títulos de subsubseções convertidos em letras maiúsculas
	% -- opções do pacote babel --
	english,			% idioma adicional para hifenização
	brazil,				% o último idioma é o principal do documento
	]{abntex2}
\usepackage{tabularx}
% ---
% PACOTES
% ---

% ---
% Pacotes fundamentais 
% ---
\usepackage{lmodern}			% Usa a fonte Latin Modern
\usepackage[T1]{fontenc}		% Selecao de codigos de fonte.
\usepackage[utf8]{inputenc}		% Codificacao do documento (conversão automática dos acentos)
\usepackage{indentfirst}		% Indenta o primeiro parágrafo de cada seção.
\usepackage{color}				% Controle das cores
\usepackage{graphicx}			% Inclusão de gráficos
\usepackage{microtype} 			% para melhorias de justificação
% ---

%--------------Pacotes de citações
\usepackage[brazilian,hyperpageref]{backref}	 % Paginas com as citações na bibl
\usepackage[alf]{abntex2cite}	% Citações padrão ABNT

\usepackage{hyperref}

% --- 
% CONFIGURAÇÕES DE PACOTES
% --- 

% ---
% Configurações do pacote backref
% Usado sem a opção hyperpageref de backref
\renewcommand{\backrefpagesname}{Citado na(s) página(s):~}
% Texto padrão antes do número das páginas
\renewcommand{\backref}{}
% Define os textos da citação
\renewcommand*{\backrefalt}[4]{
	\ifcase #1 %
		Nenhuma citação no texto.%
	\or
		Citado na página #2.%
	\else
		Citado #1 vezes nas páginas #2.%
	\fi}%
% ---

% alterando o aspecto da cor azul
\definecolor{blue}{RGB}{41,5,195}

% informações do PDF
\makeatletter
\hypersetup{
     	%pagebackref=true,
		pdftitle={\@title}, 
		pdfauthor={\@author},
    	pdfsubject={\imprimirpreambulo},
	    pdfcreator={LaTeX with abnTeX2},
		pdfkeywords={abnt}{latex}{abntex}{abntex2}{plano de trabalho}, 
		colorlinks=true,       		% false: boxed links; true: colored links
    	linkcolor=blue,          	% color of internal links
    	citecolor=blue,        		% color of links to bibliography
    	filecolor=magenta,      		% color of file links
		urlcolor=blue,
		bookmarksdepth=4
}
\makeatother
% --- 

% --- 
% Espaçamentos entre linhas e parágrafos 
% --- 

% O tamanho do parágrafo é dado por:
\setlength{\parindent}{1.3cm}

% Controle do espaçamento entre um parágrafo e outro:
\setlength{\parskip}{0.2cm}  % tente também \onelineskip

% ---
% compila o indice
% ---
\makeindex
% ---

%-----------------------------------------
\begin{document}

% Retira espaço extra obsoleto entre as frases.
\frenchspacing 

\textbf{Graduação em Ciência da Computação - UFU}

\textbf{Disciplina:} GBC235 - Tópicos especiais de Segurança da Informação - Laboratórios e outras atividades práticas sobre Segurança de Redes 

\textbf{Professor:} Prof. Rodrigo Sanches Miani

\textbf{Nome:} João Paulo de Oliveira

% ------------------ Título -----------------------
\section*{\centerline{\Large \textbf{Relatório XYZ}}}
% -------------------------------------------------

%------------------- Corpo de dados --------------
\vspace*{0.5cm}

\textbf{Objetivo da atividade:} apresentar sucintamente o objetivo da atividade.

\textbf{Questão 1)}
Amafil
Correios
Apple"

\textbf{Questão 2)}


\begin{center}
    \begin{table}[!htb]
    \begin{tabular}{|l|l|}
    \hline
    Endereço IP & Info \\ \hline
     &  \\ \hline
    \end{tabular}
    \end{table}
\end{center}

\textbf{Questão 3)}
A diretiva site:ufu.br inurl:admin é uma tentativa de encontrar alguma url que esteja sob o domínio de administradores, ou tenha relação com tais

\begin{table}[h]
\begin{tabular}{|l|l|}
\hline
Endereço IP & Info \\ \hline
187.17.111.103 (www.amafil.com.br) & \begin{tabular}[c]{@{}l@{}}Feito em wordpress, pagina de login:   http://www.amafil.com.br/wp-login.php\\ Hospedado na UOL, localizado em SP, segundo o https://www.iplocation.net/\\ \\ Nada encontrado nas diretivas\end{tabular} \\ \hline
201.48.198.80 (www.correios.com.br) & Nada encontrado nas diretivas, apenas referências para acesso, e a página de login: https://ediac.correios.com.br/ediweb/ \\ \hline
95.100.45.144 (www.apple.com) & Nada encontrado nas diretivas, apenas manuais, exemplos de configuração, notícias, etc \\ \hline
\end{tabular}
\end{table}

\textbf{Questão 4)}
\bibliography{ref}

Listar as principais referências bibliográficas (livros, artigos, sites) utilizadas ao redigir o relatório.

\end{document}